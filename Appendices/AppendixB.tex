% Appendix B

\chapter{Instalación de Entornos de desarrollo} % Main appendix title

\label{AppendixB} % For referencing this appendix elsewhere, use \ref{AppendixA}

\lhead{\textit{Kinetis Design Studio}}  % This is for the header on each page - perhaps a shortened title

\section{\textit{Kinetis Design Studio}} % This is for the header on each page - perhaps a shortened title

Software necesario:\footnote{Para la descarga de 2 y 3 se necesita registro en la página.}

\begin{enumerate}
\item Sistema Operativo GNU/Linux 64 bits.
\item Kinetis Design Studio 2.0.0. \href{url}{http://www.nxp.com/products/software-and-tools/run-time-software/kinetis-software-and-tools/ides-for-kinetis-mcus/kinetis-design-studio-integrated-development-environment-ide:KDS\_IDE}
\footnote{En el archivo descargado debe cambiarse la extensión por ``.deb" }
\item MQX 4.1.1. \href{url}{https://community.freescale.com/community/mqx}
\footnote{En el archivo descargado debe cambiarse la extensión por ``.tar.gz"}
\end{enumerate}

Instalación:\footnote{Los nombres de los archivos pueden variar.}
\begin{enumerate}
\item Instalar \textit{KDS.deb}: \\
\verb|sudo dpkg -i KDS.deb|
\item Descomprimir MQX4.1.1.tar.gz:
\verb|tar -xzvf MQX4.1.1.tar.gz|
\item Compilar liberías de MQX:
	\begin{enumerate}
	\item Abrir KDS e importar el siguiente proyecto ~/MQX/build/frdmk64f/kds/build\_libs.wsd:
	\item Compilar todos los proyectos (\textit{build all}):	
		\begin{itemize}
		\item BSP
		\item PSP
		\item RTCS
		\item USB
		\item Shell
		\end{itemize}
	Se generarán las librerías necesarias para utilizar el Sistema Operativo	MQX.
	\item Importar un proyecto de ejemplo.	
	\end{enumerate}
\end{enumerate}
Para generar un nuevo proyecto, se debe copiar un ejemplo incluído en el SO y modificar los \textit{PATHs} de las librerías ya compiladas para que apunten correctamente. Esto se debe a que la dirección guardada en el IDE está dirigida de forma relativa del proyecto original. 



\newpage
\lhead{Apéndice B2. \textit{Rails}}  % This is for the header on each page - perhaps a shortened title

\section{\textit{Rails}} % This is for the header on each page - perhaps a shortened title

Software necesario:
\begin{enumerate}
\item Sistema Operativo GNU/Linux 64 bits.
\item Rails.
\end{enumerate}
Nota:
\begin{quotation}
Para instalar el \textit{framework Rails} y Ruby , no es recomendable utilizar los paquetes del gestor de descargas del sistema operativo por cuestiones de compatibilidad.\\
\end{quotation}

Pasos para la instalación(siempre desde la consola):

\begin{enumerate}
\item Actualizar sistema:\\ 
\verb|sudo apt-get update|
\item Instalar Curl:\\ 
\verb|sudo apt-get install curl|

\item Instalar Ruby usando \textit{RVM}
	\begin{enumerate}
	\item Instalar RVM:\\
	\verb|\curl -L https://get.rvm.io | bash -s stable --ruby |
	\item Instalar Ruby:
		\begin{enumerate}
		\item \verb|rvm get stable --autolibs=enable |
		\item \verb|rvm install ruby |
		\item \verb|rvm --default use ruby-2.3.0 |
		\end{enumerate}
	\end{enumerate}

\item Chequear el \textit{Gem Manager}:\\
\verb|gem -v|
\item Actualizar Gemas:\\
\verb|gem update --system|
\item Instalar gema Bundler:\\
\verb|gem install bundler|
\item Crear \textit{workspace}:\\
\verb|mkdir workspace|
\end{enumerate}
\newpage


\lhead{Apéndice B3. \emph{Android Studio}} % This is for the header on each page - perhaps a shortened title
\section{\textit{Android Studio}} % This is for the header on each page - perhaps a shortened title
Software necesario:
\begin{itemize}
\item Sistema Operativo GNU/Linux 64 bits.
\item Java \textit{JDK}.
\item \textit{Android Studio}. \href{url}{http://developer.android.com/intl/es/sdk/index.html\#top}
\end{itemize}

Pasos para la instalación:
\begin{enumerate}
\item Descargar e instalar el \textit{Java Development Kit(JDK)}, puede ser la versión libre OpenJDK(desde consola):
\verb|sudo apt-get intall openjdk-7-jdk|
\item Descargar e intalar \textit{Android Studio} desde el \textit{link} anterior.
\end{enumerate}

